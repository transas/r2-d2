% !TEX TS-program = pdflatex
% !TEX encoding = UTF-8 Unicode

% This is a simple template for a LaTeX document using the "article" class.
% See "book", "report", "letter" for other types of document.

\documentclass[11pt]{book} % use larger type; default would be 10pt

\usepackage[utf8]{inputenc} % set input encoding (not needed with XeLaTeX)
\usepackage[russian]{babel}

%%% Examples of Article customizations
% These packages are optional, depending whether you want the features they provide.
% See the LaTeX Companion or other references for full information.

%%% PAGE DIMENSIONS
\usepackage{geometry} % to change the page dimensions
\geometry{a4paper} % or letterpaper (US) or a5paper or....
% \geometry{margin=2in} % for example, change the margins to 2 inches all round
% \geometry{landscape} % set up the page for landscape
%   read geometry.pdf for detailed page layout information

\usepackage{graphicx} % support the \includegraphics command and options

% \usepackage[parfill]{parskip} % Activate to begin paragraphs with an empty line rather than an indent
\usepackage{indentfirst}

\parskip=10pt

%%% PACKAGES
%\usepackage{booktabs} % for much better looking tables
\usepackage{array} % for better arrays (eg matrices) in maths
\usepackage{paralist} % very flexible & customisable lists (eg. enumerate/itemize, etc.)
\usepackage{verbatim} % adds environment for commenting out blocks of text & for better verbatim
\usepackage{subfig} % make it possible to include more than one captioned figure/table in a single float
% These packages are all incorporated in the memoir class to one degree or another...

%%% HEADERS & FOOTERS
\usepackage{fancyhdr} % This should be set AFTER setting up the page geometry
\pagestyle{fancy} % options: empty , plain , fancy
\renewcommand{\headrulewidth}{0pt} % customise the layout...
\lhead{}\chead{}\rhead{}
\lfoot{}\cfoot{\thepage}\rfoot{}

%%% SECTION TITLE APPEARANCE
\usepackage{sectsty}
%\allsectionsfont{\sffamily\mdseries\upshape} % (See the fntguide.pdf for font help)
% (This matches ConTeXt defaults)

%%% ToC (table of contents) APPEARANCE
\usepackage[nottoc,notlof,notlot]{tocbibind} % Put the bibliography in the ToC
\usepackage[titles,subfigure]{tocloft} % Alter the style of the Table of Contents
%\renewcommand{\cftsecfont}{\rmfamily\mdseries\upshape}
%\renewcommand{\cftsecpagefont}{\rmfamily\mdseries\upshape} % No bold!

%%% END Article customizations

%%% The "real" document content comes below...

\title{R2D2.ironbot: Руководство пользователя}
\author{И. Меркин}
%\date{} % Activate to display a given date or no date (if empty),
         % otherwise the current date is printed 

\begin{document}
\maketitle

\tableofcontents

\chapter{Базовые сведения}

R2D2.ironbot входит в набор модулей R2D2 и является модулем расширения для фреймворка robot. Модуль предоставляет функциональность для автоматизации традиционных интерфейсов пользователя под Windows. При этом в качестве средства работы с пользовательскими интерфейсами выступает библиотека White (.Net), вследствие чего предполагается запуск фреймворка robot не из-под CPython, а из-под IronPython.

\section{Инсталляция и включение в проект}

Здесь будет интересный и увлекательный рассказ про стучание головой о клавиатуру в процессе инсталляции. Надо написать про:
Python, wxWidgets, RIDE, IronPython, lxml, log4net, White, Castle.Core.


\section{Ключевые слова и их параметры: общий подход}

\verb|R2D2.ironbot| реализует множество ключевых слов \verb|Robot|, обеспечивающих запуск и контроль работы приложений, поиск элементов пользовательского интерфейса и управление ими.
В основу дизайна этих ключевых слов положены два общих правила. Первое касается именования самих ключевых слов. Все ключевые слова делятся на несколько групп, и имя каждого ключевого слова всегда начинается с имени группы. Так, группа \verb"App" обеспечивает работу с приложениями (запуск, контроль выполнения и т.п.), а группа \verb"Wnd" --- с окнами. Соответственно, типичное название ключевого слова из группы \verb"App" --- \verb"App Launch" (сначала группа, потом характер действия), а из группы \verb"Wnd"  --- \verb"Wnd Close".

Второе правило касается набора параметров, принимаемых ключевыми словами. В соответствии с принятыми соглашениями, ключевые слова R2D2.ironbot имеют два вида параметров --- позиционные и именованые. Позиционные параметры всегда идут первыми, причём в строго фиксированной последовательности, и представляют собой традиционные параметры ключевых слов robot. Именованые параметры следуют после позиционных, обычно в произвольной последовательности. Каждый именованый параметр может представлять собой конструкцию из одного или более параметров ключевого слова robot, первый из которых --- имя параметра, а последующие --- набор значений параметра (допускаются именованые параметры с произвольным количеством значений; параметр, не имеющий значений, называется далее в описании \emph{флагом}, т.к. для него важен только сам факт присутствия).

Рассмотрим примеры задания параметров на типичном ключевом слове. Ниже мы видим ключевое слово \verb"App Launch", предназначенное для запуска приложений:

\begin{verbatim}
App Launch    ipy.exe    params    ../winforms.py    test_teardown
\end{verbatim}

Здесь присутствует один позиционный параметр, путь к файлу приложения. Этот параметр является обязательным, без него ключевое слово не имело бы смысла, вследствие чего он и был сделан позиционным. На этом позиционные параметры кончаются, далее следуют именованые.

Именованый параметр \verb"params" имеет одно значение, и это значение --- строка аргументов, которая будет передана приложению при запуске (\verb"../winforms.py"). Параметр не является обязательным. Обратите внимание, что в данном случае один именованый параметр представлен двумя параметрами ключевого слова robot. В популярной среде разработки RIDE, где ключевые слова и их параметры представляются ячейками таблицы, этот именованый параметр займёт две ячейки.

Именованый параметр \verb"test_teardown" является типичным \emph"флагом". За ним не идёт никакое дополнительное значение, важен сам факт присутствия флага среди именованых параметров. В данном случае параметр означает, что в том случае, если до конца теста (test teardown --- термин, означающий код, выполняемый при деинициализации теста) приложение не завершится естественным образом, то при деинициализации теста приложение должно быть аварийно завершено.

Характерно, что ту же самую команду можно записать и следующим образом, переставив именованые параметры между собой:

\begin{verbatim}
App Launch    ipy.exe    test_teardown    params    ../winforms.py
\end{verbatim}


\section{Спецификация времени ожидания}

В тех случаях, когда в значении параметра ключевого слова следует указать время ожидания, существуют два способа это сделать. Во-первых, можно указать точное время ожидания, не зависящее от производительности компьютера. Это делается путём задания в качестве значения параметра вещественного числа с постфиксом \verb"s" (секунды), \verb"ms" (миллисекунды), \verb"m" (минуты), \verb"h" (часы). Во-вторых, можно указать время, скорректированное с учётом оценки производительности компьютера (оценка производительности производится с помощью ключевых слов группы \verb"Benchmark", см. описание слов этой группы), т.е. на медленном компьютере в действительности пройдёт больше времени, а на быстром --- меньше. Для этого к спецификации времени в начале приписывается символ \verb"~". Кроме того, можно описать бесконечную задержку, указав в качестве спецификации задержки слово \verb"forever".

Вот примеры корректных спецификаций времени: \verb"10s", \verb"0.5s", \verb"10ms", \verb"~10s", \verb"~0.5m", \verb"~10h",  \verb"forever".

\section{Наиболее распространённые именованые параметры}
Рассматриваемые ниже именованые параметры применимы ко многим ключевым словам. Эти параметры можно разделить на три группы: параметры, задающие критерии успешности поиска или фильтрации; параемтр, задающий время ожидания; параметр, заставляющий ключевое слово потерпеть неудачу, если не удалось дождаться требуемого другими параметрами состояния объекта или объектов.

\subsection{Ожидание}

В тех случаях, когда ключевое слово ждёт желаемого результата (открытие окна или окон, появление элементов пользовательского интерфейса, старт или завершение приложения), оно всегда может принимать именованый параметр \verb|timeout|, значением которого является спецификация времени ожидания. Наличие этого параметра заставляет ключевое слово изменить свою работу. Если без него выполняется однократная проверка состояния объекта или объектов и возврат результата, соответствующего итогам проверки, то добавление параметра \verb|timeout| заставляет ключевое слово повторять попытки либо до тех пор, когда результат будет соответствовать ожидаемому, либо до тех пор, пока величина тайм-аута истечёт (что наступит раньше).

Например, следующее ключевое слово будет ждать завершения приложения в течение не более чем 5 секунд, вернув \verb|True|, если дождётся, и \verb|False|, если нет:
\begin{verbatim}
App State    ${app}    not_running    timeout    5s
\end{verbatim}

Если бы тайм-аут не был указан, ключевое слово вернуло бы результат после первой же попытки. Для не ограниченного по времени ожидания используется \verb|timeout    forever|.

Кроме параметра \verb|timeout| есть и другие случаи, когда используется спецификация времени ожидания. Они упоминаются в описании соответстующих ключевых слов.


\subsection{Критерии успешности поиска или фильтрации}
Подавляющее большинство ключевых слов, выполняющих поиск, фильтрацию и контроль за состоянием объектов (поиск окон и контролов, управление состоянием приложений и т.п.), помимо ожидания, поддерживает несколько именованых параметров, позволяющих очень широко варьировать действие ключевого слова. Таки ключевые слова обычно могут работать не только с отдельными объектами, но и со списками объектов (например, ключевое слово \verb|App State| работает как с одним приложением, так и со списком приложений). При этом операция выполняется с каждым приложением из списка. Важной особенностью работы со списком является то, что состояния объектов могут не быть одинаковыми (допустим, в случае \verb|App State| только часть приложений может завершиться). А в случае операций поиска найденное количество объектов может изменяться со временем (например, поиск окон работает именно так, если окна открываются или закрываются во время его работы).

В то же время, понятие <<ожидаемый результат>> может быть различно в зависимости от потребностей пользователя. Иногда нужно дождаться, пока хотя бы одно из приложений завершится. Иногда нужно дождаться открытия ровно двух окон, подпадающих под запрос поиска, а иногда надо убедится, что ни один объект не подпадает под запрос. Для реализации такой функциональности существует группа именованых параметров, которая включает:
\begin{itemize}
\item \verb|all| (флаг) --- все объекты должны достичь требуемого состояния;
\item \verb|any| (флаг) --- любой из объектов должен достичь требуемого состояния;
\item \verb|single| (флаг) --- ровно один из объектов должен достичь требуемого состояния, результат операции --- не список, а одиночный объект либо \verb|None| (допустимо комбинирование с \verb|none|, означающее <<ноль или один>>);
\item \verb|none| (флаг) --- ровно один из объектов должен достичь требуемого состояния, результат операции --- булевский (при комбинировании с \verb|single| -- объект либо \verb|None|);
\item \verb|number| (значение --- целое число) --- требуемого состояния должны достичь ровно столько объектов, сколько указано значением параметра (результат --- список).
\end{itemize}

\subsection{Атрибуты объектов и операции над ними}

Библиотека R2D2.Ironbot манипулирует значительным количеством разнообразных объектов, в основном связанных с пользовательским интерфейсом. В принятом в данной библиотеке понимании, объекты обладают атрибутами, и различные ключевые слова библиотеки так или иначе с этими атрибутами имеют дело. Атрибуты используются при фильтрации искомых объектов, при выполнении над объектами различных действий, при получении сведений о них и т.п. Над атрибутами производятся различные операции (например, \verb|get| --- взятие значения, \verb|set| --- запись значения, \verb|wait| --- фильтрация путём сравнения знаения с образцом, которую можно использовать для ожидания, \verb|do| --- выполнение атрибута-действия). Не каждый атрибут поддерживает все возможные операции, так, есть атрибуты только для чтения, атрибуты для чтения и записи, наконец, атрибуты-действия, например, атрибут \verb|click|, поддерживающий единственную операцию \verb|do|, которая выполняет щелчок мыши на объекте.

Когда в параметрах ключевого слова упоминаются атрибуты, указываются имя действия и имя атрибута, разделённые пробелом, за которыми могут идти дополнительные значения:

\begin{verbatim}
do click
set idx_selected    5
get id
wait automation_id    MY_AID
\end{verbatim}

В некоторых случаях допускается не указывать действие, а только имя атрибута. Это значит, что используется то действие, которое наиболее характерно для выполняемого ключевого слова. Так, ключевые слова поиска обычно позволяют не указывать действие, т.к. это действие --- всегда \verb|wait|. Для ключевых слов, читающих и записывающих значения атрибутов, действием по умолчанию является \verb|get|.

В тех случаях, когда в одном ключевом слове указывается множество операций с атрибутами, сначала выполняется ожидание (\verb|wait|), потом взятие значений (\verb|get|, внутри списка действий \verb|get| порядок выполнения определяется по алфавитному порядку имён атрибутов), потом действия \verb|do|, наконец, \verb|set|. При этом тот порядок, в котором действия перечислены в параметрах ключевого слова, не играет никакой роли и не определяет порядка выполнения.

Пример, приводимый ниже --- команда работы с атрибутами элемента GUI (\verb|list|). Ждёт не долее 5 секунд момента, когда в списке будет ровно шесть объектов, а сам список будет доступен для редактирования (enabled), после чего выбирает в нём элемент с индексом 1.
\begin{verbatim}
Ctl Attr    ${list}    wait num_items    6    wait enabled 
... set idx_selected    1    timeout    5s
\end{verbatim}


\subsubsection{Атрибуты процессов}

Атрибуты процессов используются командами группы \verb|Proc|, точнее --- командой \verb|Proc Filter|. Для всех атрибутов процессов единственной возможной операцией является \verb|wait|.

\verb|id| --- Id процесса. Единственная операция --- \verb|wait|, параметр --- значение \verb"Id" для искомого процесса.

\verb|title| --- заголовок главного окна. Единственная операция --- \verb|wait|, параметр --- значение заголовка для главного окна искомого процесса.

\verb|re_title| --- заголовок главного окна для поиска по регулярному выражению. Единственная операция --- \verb|wait|, параметр --- регулярное выражение для сравнения (\verb|match|) заголовка для главного окна искомого процесса.

\verb|name| --- имя процесса. Единственная операция --- \verb|wait|, параметр --- значение имени искомого процесса.

\verb|re_name| --- имя процесса для поиска по регулярному выражению. Единственная операция --- \verb|wait|, параметр --- регулярное выражение для сравнения (\verb|match|) --- имени искомого процесса.
		



\subsubsection{Атрибуты окон}

Атрибуты окон используются командами \verb"Wnd Attr" (действия \verb|get|, \verb|set| и \verb|do|), а также \verb|Wnd Get| и \verb|Wnd Filter| (только действие \verb|wait|).

\verb|keyboard| --- объект <<клавиатура>>>, связанный с окном. Только взятие действием \verb|get|.

\verb'id' --- Id окна, действия \verb|wait| и \verb|get|.

\verb're_id' --- Id окна для поиска по регулярному выражению, действие \verb|wait|.

\verb'title' --- заголовок окна, действия \verb|wait| и \verb|get|.

\verb're_title' --- заголовок окна для поиска по регулярному выражению, действие \verb|wait|.

\verb'automation_id' --- AutomationId окна, действия \verb|wait| и \verb|get|.

\verb're_automation_id' --- AutomationId окна для поиска по регулярному выражению, действие \verb|wait|.


\verb'closed' --- окно минимизировано (булевское значение), соответствует атрибуту \verb|IsClosed| в объекте библиотеки \verb|White| (почему этот атрибут так назван, ясно не вполне), действия \verb|wait| и \verb|get|.

\verb'not_closed' --- инверсия предыдущего атрибута (\verb|closed|).

\verb'active' --- окно активно (булевское значение), соответствует атрибуту \verb|IsActive| в объекте библиотеки \verb|White|, действия \verb|wait| и \verb|get|.

\verb'not_active' --- инверсия предыдущего атрибута (\verb|active|).

\verb'close' --- атрибут для закрытия окна, действие \verb|do|.

\verb'wait_while_busy' --- атрибут для ожидания в течение периода неактивнсти окна (т.е. до тех пор, пока окно не будет готово обрабатывать внешние воздействия), действие \verb|do|.

\verb|menu| --- атрибут для получения меню окна, действие \verb|get|.


\verb'texts' --- атрибут для получения текста окна (текст включает контролы Label и Button), действие \verb|get| возвращает список строк для контролов.


\verb'in_texts' --- атрибут для поиска/фильтрации окон, в текстах которых присутствует заданная строка, действие \verb|wait|, строка указывается парамером.
 
\verb're_in_texts' --- атрибут для поиска/фильтрации окон, в текстах которых присутствует строка, подходящая (\verb|match|) под регулярное выражение, действие \verb|wait|, выражение указывается парамером.

\verb'merged_texts' --- атрибут для получения текста окна (текст включает контролы Label и Button), действие \verb|get| возвращает все строки объединёнными в одну, между объединяемыми строками вставляется \verb'\\n'.


\subsubsection{Атрибуты элементов GUI (контролов)}


\verb'id' --- Id элемента, действия \verb|wait| (значение --- параметр) и \verb|get| (без параметров).

\verb're_id' --- Id элемента для поиска по регулярному выражению, действие \verb|wait| (выражение --- параметр).

\verb'name' --- имя элемента (\verb|Name| в терминах \verb|White|), действия \verb|wait| (значение --- параметр) и \verb|get| (без параметров).

\verb're_name' --- имя элемента для поиска по регулярному выражению, действия \verb|wait| (выражение --- параметр) и \verb|get| (без параметров).

\verb'automation_id' --- AutomationId, действия \verb|wait| (выражение --- параметр) и \verb|get|.

\verb're_automation_id' --- AutomationId окна для поиска по регулярному выражению, действие \verb|wait|.

\verb'enabled' и \verb'disabled' --- пара булевских атрибутов (всегда противоположных) для проверки, разрешён ли соответствующий элемент GUI, действия \verb|wait| (без параметра) и \verb|get|.

\verb'click', \verb'dclick' и \verb|rclick| --- атрибуты для выполнения действий мышью (щелчок, двойной щелчок и щелчок правой кнопкой соответственно), действия \verb|do| и \verb|get| (оба действия равнозначны, перемещают мышку в центр контрола и выполняют соответствующий вид щелчка).

\verb|visible| --- булевский атрибут, поддерживающий действие \verb|get| и соответствующий вызову метода \verb|Visible()| у объекта библиотеки \verb|White|.

\verb|focus| --- булевский атрибут, поддерживающий действие \verb|get| (сообщает в виде булевского значения, имеет ли элемент фокус) и действие \verb|do| (передаёт фокус).

Дальнейшие атрибуты являются специфичными для отдельных типов контролов.

\verb|text| --- текстовое содержимое контрола (у \verb|TextBox|, \verb|ListItem|, \verb|TreeNode|), действие \verb|get| у каждого из них, действие \verb|set| (с параметром, задающим значение) --- только у \verb|TextBox|.


\verb|checked| и \verb|unchecked| --- пара зависимых булевских атрибутов, противоположных друг другу. Действия \verb|get| и \verb|set| (особенность действия \verb|set| в том, что оно принимает булевский параметр --- строку \verb|false| или \verb|true|). Поддерживаемые элементы интерфейса: \verb|CheckBox|, \verb|RadioButton|, \verb|ListItem| (последний --- если в списке есть чекбоксы).

Атрибуты деревьев (\verb|Tree|) и их узлов (\verb|TreeNode|):

\verb|nodes| --- атрибут для получения узлов верхнего уровня для дерева и прямых 

\verb|node_by_path| --- взятие в дереве (но не в узле) одного из узлов. Действие \verb|get|, параметром действию служит путь в дереве. Путь представляет собой последовательность строк (имён узлов в пути), каждое имя кодируется одним параметром ключевого слова \verb|Robot|. Путь завершается либо значением \verb|<END>|, либо концом списка параметров.

\verb|selected_node| --- атрибут для взятия выбранного узла у дерева действием \verb|get|.

\verb|selected| --- булевский атрибут для проверки того, выбран ли узел, и для выбора узла. Действие \verb|get| выдаёт булевское значение. Действие \verb|set| принимает булевское значение (строка \verb|true|/\verb|false|).


\verb|collapse| и \verb|expand| --- пара атрибутов для закрытия и раскрытия узлов дерева (есть только у узлов, но не у дерева в целом). Действие \verb|do|.

Атрибуты для работы с меню и подобными объектами (\verb|MenuBar|, \verb|ToolStrip|):

\verb|menuitem| --- атрибут для получения объекта, соответствующего заданному элементу меню (элемент задаётся путём), а также для выполнения клика по нему. Действие \verb|get| получает в качестве параметра путь в меню, полностью аналогичный пути в дереве (последовательность параметров с именами, завершаемая либо значением \verb|<END>| либо концом списка параметров) и возвращает объект, соответствующий элементу меню. Действие \verb|click| схоже с \verb|get|, но выполняет клик на соответствующем элементе.

Атрибуты для работы со списками (\verb|ListBox|):

\verb|selected| --- атрибут для получения и установки выбранного элемента. Действие \verb|get| возвращает объект, соответствующий выбранному в списке элементу. Действие \verb|set| принимает параметр (по выбору пользователя, либо объект, либо целочисленный индекс в списке) и выбирает его.

\verb|idx_selected| --- атрибут почти аналогичный \verb|selected|, но действие \verb|get| возвращает индекс элемента списка, а действие \verb|set| принимает только индекс, но не объект.

\verb|num_items| --- атрибут, позволяющий получать число элементов в списке (действие \verb|get|) и ждать, пока число элементов в списке не станет желаемым (действие \verb|wait|, параметром принимает число; применяется для ожидания момента, когда список будет заполнен, если заполнение происходит динамически).

\verb|listitems| --- атрибут, действие \verb|get| над которрым возвращает полный список элементов списка.

Атрибуты для работы со страничными закладками (\verb|Tab|):

\verb|tabpages| --- атрибут для получения списка страниц, действие \verb|get| возвращает список объектов, соответствующих страницам.

\verb|num_tabpages| --- атрибут для получения числа страниц, действие \verb|get| возвращает число объектов, соответствующих страницам.

\verb|idx_selected| и \verb|name_selected| --- атрибуты для получения и задания соответственно индекса или имени выбранной страницы. Действие \verb|get| возвращает индекс или имя, действие \verb|set| принимает индекс или имя в качестве параметра.


\subsubsection{Атрибуты объекта <<клавиатура>> и обозначения клавиш}

Объект <<клавиатура>> (доступен как атрибут \verb|keyboard| у окна) применяется для эмуляции клавиатурного ввода. Все его атрибуты имеют одно действие \verb|do| с одним параметром, либо обозначающим имя клавиши, либо задающим строку, которая вводится с клавиатуры.

Имена клавиш делятся на специальные (\verb|F1|, \verb|ALT| и т.п.) и стандартные (\verb|A|, \verb|1| и т.п.). Список имён специальных клавиш: \verb'ALT', \verb'BACKSPACE', \verb'CAPS', \verb'CONTROL', \verb'DELETE', \verb'DOWN', \verb'END', \verb'ESCAPE', \verb'F1'--\verb'F24', \verb'HOME', \verb'INSERT', \verb'LEFT', \verb'LEFT_ALT', \verb'LWIN', \verb'NUMLOCK', \verb'PAGEDOWN', \verb'PAGEUP', \verb'PRINT', \verb'PRINTSCREEN', \verb'RETURN', \verb'RIGHT', \verb'RIGHT_ALT', \verb'RWIN', \verb'SCROLL', \verb'SHIFT', \verb'SPACE', \verb'TAB'.


Атрибуты:

\verb|hold| --- нажать клавишу (и не отпускать).

\verb|leave| --- отпустить клавишу.

\verb|press| --- нажать и отпустить клавишу.

\verb|enter| --- ввести с клавиатуры строку.

\subsection{Строгий режим команд ожидания и проверки состояния}

Включается параметром-флагом \verb|assert|, присутствие которого всегда приводит к одному и тому же результату, не зависящему от ключевого слова: если ключевое слово не смогло дождаться требуемого состояния объектов, с которыми работает (на само требуемое состояние влияют как рассмотренные выше критерии успешности, так и многие специфические для отдельных ключевых слов параметры), оно не возвращает результат, а выбрасывает исключение, которое, если в коде на языке \verb|Robot| не принято специальных мер, приводит к тому, что тест терпит неудачу.

\chapter{Описание ключевых слов}


\section{Мониторинг ошибок и управление им}

Мониторинг ошибок --- функция библиотеки, предназначенная для выявления и фиксации внезапно возникающих ошибок тестируемых программ. Под такими ошибками понимаются в первую очередь крэши. Ключевые слова данной группы позволяют устанавливать специальные обработчики ошибок. Каждый такой обработчик представляет собой процедуру, оформленную в виде отдельного теста в вормате Robot Framework и выполняющую следующие действия:

\begin{itemize}
\item длительное ожидание появления окна ошибки;
\item при появлении окна --- действия по его закрытию, затем завершение теста с неудачным результатом. 
\end{itemize}

Обычно такие тесты-обработчики оформляются в виде отдельного файла с набором тестов. Каждый тест из набора в первый раз запускается в момент инициализации мониторинга, а при нормальном завершении кода обработчика всегда производится его автоматический перезапуск. При этом, если ошибок не выявляется, обработчики работают до конца пользовательского теста либо до момента явной остановки процесса мониторинга. Если же ошибки выявляются, выполняется заключительная процедура закрытия окон с ошибками, после чего пользовательский тест завершается неудачей.

Если ошибки выявляются во время работы теста, это приводит к неудаче теста на ближайшем ключевом слове библиотеки \verb|Ironbot|, связанном с ожиданием, либо на ключевом слове \verb|Finalize Monitors|. Именно поэтому, во-первых, настоятельно рекомендуется явное выполнение \verb|Finalize Monitors| в конце теста (процессы обработчиков будут завершены и без этого, ошибка может быть выявлена лишь на этапе teardown, в результате чего тест уже не сможет завершиться неудачей), во-вторых, вместо стандартного ключевого слова \verb|Sleep| рекомендуется использование ключевое слово \verb|Dream|, которое не только выдерживает паузу, но и осуществляет в процессе ожидания все необходимые проверки. 



\subsection{Setup Monitors (установка мониторинга)}
Это ключевое слово инициализирует процесс мониторинга, запуская перечисляемые пользователем обработчики из заданного файла с набором тестов. Наряду с набором обработчиков задаются два интервала времени (см. описание позиционных параметров), которые влияют на процесс завершения обработки ошибок в том случае, если ошибка произошла.

\subsubsection*{Название} 
\verb"Setup Monitors"

\subsubsection*{Позиционные параметры} 
\verb"delay" --- максимальная задержка между появлением окон ошибок при завершении процесса обработки. При выявлении одной ошибки процесс мониторинга продолжает работать в течение какого-то времени, чтобы обеспечить корректное закрытие окон ошибок, которые могут возникнуть впоследствии (множественные ошибки при крэше являются довольно частой ситуацией, а тестируемое приложение обычно хочется завершить полностью). Такое время ожидания ошибки и задаётся данным параметром. После каждой новой выявленной ошибки отсчёт времени начинается заранее, то есть смысл параметра --- максимальный временной интервал между ошибками в серии, при котором гарантируется обработка всей серии. В связи с тем, что происходящий при обработке повторяющихся ошибок регулярный перезапуск обработчиков иногда может быть не слишком быстрым, рекомендуется устанавливать этот параметр хотя бы в десятки секунд, чтобы избежать потенциальных проблем.

\verb"total_delay" --- полное максимальное время обработки всех ошибок после того, как возникла первая. По истечении этого времени пользовательский тест завершается неудачей независимо от того, возникали ли новые ошибки. Смысл этого параметра --- создание возможности избежать бесконечного зацикливания процесса ожидания новых ошибок, если сами ошибки возникают бесконечно. Рекомендуется использовать достаточно большое значение, например, несколько минут (а иногда, возможно, и часов).

\verb"test_suite" --- файл с набором тестов, из которого следует брать обработчики.

\verb"*tests" --- серия параметров, задающих имена тестов в наборе. Все перечисляемые тесты будут запущены как обработчики, каждый в своём процессе. Количество тестов ограничено только здравым смыслом и ресурсами комьютера.


\subsubsection*{Именованые параметры} 
Нет


\subsubsection*{Возвращаемое значение} 
Нет

\subsubsection*{Примеры}
\begin{verbatim}Setup Monitors    10s    3m    Errmon.txt    WPF_KILLER    WINFORMS_KILLER\end{verbatim}

Команда устанавливает мониторинг с использованием обработчиков \verb|WPF_KILLER| и \verb|WINFORMS_KILLER| из файла \verb|Errmon.txt|. Если после возникновения ошибки новых ошибок не возникает в течение 10 секунд, процесс обработки завершается. Максимально возможное общее время ожидания завершения обработки всех ошибок после возникновения первой --- 3 минуты. 






\subsection{Finalize Monitors (деинициализация мониторинга мониторинга)}
    Это ключевое слово без параметров завершает мониторинг ошибок. Само оно завершается ошибкой в том случае, если в процессе мониторинга были выявлены ошибки. Несмотря на то, что в ряде ситуаций использование этого слова не является обязательным (процессы обработчиков будут автоматически завершены в конце каждого теста), оно настоятельно рекомендуется к применению в конце любого теста, пользующегося механизмом мониторинга ошибок, т.к. позволяет выявлять и дублировать в результатах пользовательского теста ошибки, найденные обработчиками уже после выполнения последнего ключевого слова библиотеки \verb|Ironbot|, реализующего ожидания.

\subsubsection*{Название} 
\verb"Finalize Monitors"

\subsubsection*{Позиционные параметры} 
Нет

\subsubsection*{Именованые параметры} 
Нет

\subsubsection*{Возвращаемое значение} 
Нет

\subsubsection*{Примеры}
\begin{verbatim}Finalize Monitors\end{verbatim}




\subsection{Dream (ожидание в условиях включённого мониторинга)}
    Это ключевое слово без параметров завершает мониторинг ошибок. Само оно завершается ошибкой в том случае, если в процессе мониторинга были выявлены ошибки. Несмотря на то, что в ряде ситуаций использование этого слова не является обязательным (процессы обработчиков будут автоматически завершены в конце каждого теста), оно настоятельно рекомендуется к применению в конце любого теста, пользующегося механизмом мониторинга ошибок, т.к. позволяет выявлять и дублировать в результатах пользовательского теста ошибки, найденные обработчиками уже после выполнения последнего ключевого слова библиотеки \verb|Ironbot|, реализующего ожидания.

\subsubsection*{Название} 
\verb"Dream"

\subsubsection*{Позиционные параметры} 
\verb|delay| --- спецификация величины тайм-аута.

\subsubsection*{Именованые параметры} 
Нет

\subsubsection*{Возвращаемое значение} 
Нет

\subsubsection*{Примеры}
\begin{verbatim}Dream    20s\end{verbatim}

Команда выполняет ожидание в течение двадцати секунд.




\section{Группа Proc}
Ключевые слова группы \verb"Proc" позволяют получать список запущенных процессов и фильтровать его. Впоследствии возможно подключение к процессам с целью дальнейшего управления ими с помощью \verb"App Attach".

\subsection{Proc List (получение списка процессов)}
Это ключевое слово возвращает полный список объектов, описывающих процессы в системе.


\subsubsection*{Название} 
\verb"Proc List"

\subsubsection*{Позиционные параметры} 
Нет

\subsubsection*{Именованые параметры} 
Нет


\subsubsection*{Возвращаемое значение} 
Список объектов <<процесс>>.

\subsubsection*{Примеры}
\begin{verbatim}@{pl}=    Proc List\end{verbatim}



\subsection{Proc Filter (фильтрация списка процессов)}
Фильтрует список процессов, полученный с помощью \verb"Proc List", либо оставляя в нём только те процессы, которые подходят по критериям фильтрации, либо же выбрасывая их (т.е. оставляя не подходящие --- в варианте с флагом \verb"negative").



\subsubsection*{Название} 
\verb"Proc Filter"

\subsubsection*{Позиционные параметры} 

\verb"li" --- список процессов

\subsubsection*{Именованые параметры} 


\verb"negative" (флаг) --- <<фильтровать наоборот>>. Оставляет в выходном списке то, что не подходит под критерии. Кроме того, поддерживается полный набор стандартных параметров, кроме тайм-аута --- \verb"single", \verb|any|, \verb|all|, \verb|none|, \verb|number| и \verb"assert".

Кроме того, перечисляются атрибуты, по которым производится фильтрация. Для всех атрибутов единственным возможным действием (и действием по умолчанию, соответственно) является действие \verb|"wait"|. Поддерживаются атрибуты, перечисленные в разделе <<Атрибуты процессов>>.

\subsubsection*{Поддерживаемые атрибуты} 

Только операция \verb|wait| для следующих атрибутов: \verb"id", \verb"name", \verb"re_name", \verb"title" и \verb"re_title".



\subsubsection*{Возвращаемое значение} 
Список объектов <<процесс>> (обычно), если нет флага \verb"single", либо один процесс (при флаге \verb"single", если результатом является ровно один процесс), либо \verb"False" (при флаге \verb"single", если результатом является не один процесс; если указан флаг \verb"assert", вместо возвращения \verb"False" выбрасывается ошибка).

\subsubsection*{Примеры}
Оставить в списке \verb"@{pl}" только процессы с именем \verb"notepad":
\begin{verbatim}@{pl}=    Proc List
@{pl}=    Proc Filter    ${pl}    name    notepad\end{verbatim}

Оставить в списке \verb"@{pl}" только процессы не с именем \verb"notepad":
\begin{verbatim}@{pl}=    Proc Filter    ${pl}   negative    name    notepad\end{verbatim}

Найти единственный процесс с именем \verb"notepad", если не получится найти ровно один, выбросить ошибку:
\begin{verbatim}${p}=    Proc Filter    ${pl}   name    notepad    single    assert\end{verbatim}



\section{Группа App}
Ключевые слова группы \verb"App" обеспечивают работу с приложениями: их запуск, проверку состояния (выполняется/завершено), ожидание завершения или запуска и т.п.


\subsection{App Launch (запуск приложения)}
Это ключевое слово выполняет запуск приложения из файла, путь к которому задаётся его позиционным параметром \verb"executable". Поиск производится с учётом системных путей (включая переменную \verb"PATH").


\subsubsection*{Название} 
\verb"App Launch"

\subsubsection*{Позиционные параметры} 

\verb"executable" --- путь к исполняемому файлу. Поиск будет производиться по системным путям, включая \verb"PATH".

\subsubsection*{Именованые параметры} 
\verb"params" (значение --- строка) --- строка параметров запуска, передаваемых приложению. Внимание: все параметры в одной строке!

\verb"test_teardown" (флаг) --- если приложение не завершится к концу теста, его следует принудительно завершить. Не сочетается со \verb"suite_teardown".

\verb"suite_teardown" (флаг) --- если приложение не завершится к концу сюиты, его следует принудительно завершить. Не сочетается с \verb"test_teardown".

\verb"assert" (флаг) --- выполнение ключевого слова терпит неудачу, если запуск не произведён успешно.

\subsubsection*{Возвращаемое значение} 
При успехе --- объект, соответствующий приложению. При неудаче возвращается \verb"None", если не указан флаг \verb"assert". Если флаг \verb"assert" указан, выбрасывается та ошибка, которая привела к неудаче запуска.

\subsubsection*{Примеры}
Запустить \verb"ipy.exe" со строкой параметров \verb"wf.py". Если не завершится до конца теста, должно быть завершено аварийно. В случае ошибки запуска не возвращать \verb"None", а потерпеть неудачу:

\begin{verbatim}App Launch    ipy.exe    test_teardown    params    wf.py    assert\end{verbatim}




\subsection{App Attach (присоединение к приложениям)}
Получает процесс или список процессов. Возвращает приложение или список приложений. Может регистрировать приложения для завершения в конце теста или сюиты.

\subsubsection*{Название} 
\verb"App Attach"

\subsubsection*{Позиционные параметры} 

\verb"proc" --- процесс либо список процессов.

\subsubsection*{Именованые параметры} 
\verb"test_teardown" (флаг) --- если приложения не завершатся к концу теста, их следует принудительно завершить. Не сочетается со \verb"suite_teardown".

\verb"suite_teardown" (флаг) --- если приложения не завершатся к концу сюиты, их следует принудительно завершить. Не сочетается с \verb"test_teardown".

\subsubsection*{Возвращаемое значение} 
Объект, соответствующий приложению.

\subsubsection*{Примеры}
    Найти процесс с именем \verb"notepad" и присоединиться к нему, включив приложение в список приложений, которые должны быть завершены в конце теста:

\begin{verbatim}
@{pl}=    Proc List
${p}=    Proc Filter    @{pl}   name    notepad    single    assert
${app}=    App Attach   ${p}    test_teardown
\end{verbatim}



\subsection{App State (управление состоянием приложений)}
Позволяет проверять состояние приложения или списка приложений (работает/завершено), ждать нужного состояния и завершать приложения при необходимости. 


\subsubsection*{Название} 
\verb"App State"

\subsubsection*{Позиционные параметры} 

\verb"app" --- объект приложения или список таких объектов (выдаваемый \verb"App Launch" и \verb"App Attach").

\subsubsection*{Именованые параметры} 
\verb"params" (значение --- строка) --- строка параметров запуска, передаваемых приложению. Внимание: все параметры в одной строке!

\verb"running" (флаг) --- желаемым состоянием приложения (или приложений) вляется запущенное. Несовместимо с \verb"not_running" и \verb"kill"

\verb"not_running" (флаг) --- желаемым состоянием приложения вляется завершённое. Если указано ожидание, то это ожидание завершения. Результат операции положителен, если приложение завершилось. Несовместимо с \verb"running" и \verb"kill"

\verb"kill" (флаг) --- желаемым состоянием приложения вляется завершённое. Однако если приложение запущено (в случае ожидания --- не удалось дождаться выхода), его следует аварийно завершить. Несовместимо c \verb"running" и \verb"not_running".

\verb"timeout" (значение параметра --- спецификация времени ожидания) --- параметр, указывающий на необходимость ожидания.

Кроме того, поддерживается полный набор стандартных параметров, кроме тайм-аута --- \verb"single", \verb|any|, \verb|all|, \verb|none|, \verb|number| и \verb"assert". 


\subsubsection*{Возвращаемое значение}

Если на вход был передан список приложений, то возвращается список булевских значений. Однако если передан был одиночный объект, не обёрнутый в список, возвращается булевский флаг, также не обёрнутый в список. К такому же эффекту приводит параметр \verb|single| (возвращается объект или \verb|None|) и \verb|None| (возвращается булевский флаг без оборачивания в список).

\verb"assert" работает стандартным образом.

\subsubsection*{Примеры}

Проверить, завершено ли приложение, вернуть результат:
\begin{verbatim}${v}=    App State    ${app}    not_running\end{verbatim}

Убедиться, что приложение завершено, если нет, потерпеть неудачу:
\begin{verbatim}App State    ${app}    not_running    assert\end{verbatim}

Завершить приложение принудительно:
\begin{verbatim}App State    ${app}    kill\end{verbatim}

Завершить приложение принудительно, если оно само не завершится за 5 секунд, вернуть результат (удалось ли дождаться):
\begin{verbatim}App State    ${app}    timeout    5s    kill\end{verbatim}

Завершить приложение принудительно, если оно само не завершится за 5 секунд, если пришлось завершать, выбросить ошибку:
\begin{verbatim}App State    ${app}    timeout    5s    kill    assert\end{verbatim}

Если приложение не завершится за 5 секунд, выбросить ошибку:
\begin{verbatim}App State    ${app}    timeout    5s    assert    not_running\end{verbatim}


\section{Группа Wnd}
Ключевые слова группы \verb"Wnd" предоставляют средства поиска окон, ожидания из появления и закрытия, посылки окну сигнала на закрытие и т.п.


\subsection{Wnd Get (поиск окон с возможностью ожидания)}
Реализует все основные сценарии поиска окон, как верхнего уровня, так и в пределах приложения. Также позволяет выполнять ожидание открытия (или, наоборот, закрытия) окна.

\subsubsection*{Название}
\verb"Wnd Get"

\subsubsection*{Позиционные параметры}

Нет

\subsubsection*{Именованые параметры} 
\verb"app" (значение --- объект приложения) --- приложение, среди окон которого производится поиск. Если параметр не указан, поиск производится среди окон верхнего уровня.

Поддерживаются все стандартные именованые параметры --- \verb"single", \verb|any|, \verb|all|, \verb|none|, \verb|number|, \verb"assert" и \verb"timeout".


\subsubsection*{Поддерживаемые атрибуты} 

Для всех этих атрибутов поддерживается только действие \verb|wait|:

\begin{itemize}
\item \verb"active"/\verb"not_active";
\item \verb"closed"/\verb"not_closed";
\item \verb"id" и \verb"re_id";
\item \verb"automation_id" и \verb"re_automation_id";
\item \verb"title" и \verb"re_title".
\end{itemize}

\subsubsection*{Возвращаемое значение}
Если не задан один из флагов \verb"single" или \verb"none", всегда возвращает список найденных объектов <<окно>>. Если упомянутые флаги заданы, то правила таковы. \verb"none" заставляет вернуть булевское значение (\verb"True" --- окна не найдены, \verb"False" --- окна найдены). \verb"single" стремится вернуть объект <<окно>> или пустой объект \verb"None".

\verb"assert" работает стандартным образом.


\subsubsection*{Примеры}
На рабочем столе должно быть ровно одно окно верхнего уровня, заголовок которого соответствует регулярному выражению <<\verb"^.*ipy64\.exe$">>. Ждать этого не более 5 секунд, дождавшись --- вернуть объект <<окно>>, а не дождавшись --- выбросить исключение:

\begin{verbatim}
${win1}=    Wnd Get    re_title    ^.*ipy64\.exe$    single    assert
...     timeout    5s
\end{verbatim}

Ждать полного закрытия всех окон приложения \verb"${app2}" не более 5 секунд. Не дождавшись, выбросить исключение. Дождавшись, вернуть \verb"True"
\begin{verbatim}
Wnd Get    app    ${app2}    none    timeout    5s    assert
\end{verbatim}

Получить список всех окон приложения \verb"${app1}", поместить его в список \verb"@{li}". Затем вычисляется длина списка и заносится в переменную \verb"${l}". Обратите особое внимание, что при передаче списка ключевому слову \verb"Get Length" список указан не как \verb"@{li}", а как \verb"${li}", в противном случае значения, лежащие в списке, стали бы каждый отдельным параметром ключесвого слова:
\begin{verbatim}
@{li}=    Wnd Get    app    ${app1}
${l}=    Get Length    ${li}
\end{verbatim}



\subsection{Wnd Filter (фильтрация списка окон)}
Фильтрует ранее полученный список окон по заданным критериям. По парамтрам близко к \verb"Wnd Get", но, естественно, имеет позиционный параметр (исходный список окон) и не предоставлят функциональности ожидания. Дополнительный параметр по сравнение с \verb"Wnd Get" --- \verb"negative".

\subsubsection*{Название}
\verb"Wnd Filter"

\subsubsection*{Позиционные параметры} 

\verb"wlist" -- исходный список окон.

\subsubsection*{Именованые параметры} 

Поддерживается набор стандартных параметров, из которых имеют смысл \verb"single", \verb|none|, \verb|number| и \verb"assert""


\verb"negative" (флаг) --- инверсия условий поиска. Вместо обычного отбора тех элементов списка, которые соответствуют критериям поиска, из списка выбрасываются параметры, соответствующие критериям поиска. Важная деталь: сначала поиск выполняется так, как если бы данного флага не было (все заданные в параметрах критерии применяются по <<И>>, т.е. найденные на этом этапе окна удовлетворяют всем критериям). Потом результат <<инвертируется>>, т.е. в результат попадают те окна из исходного набора, которые не были отобраны по критериям.


\subsubsection*{Поддерживаемые атрибуты} 

Для всех этих атрибутов поддерживается только действие \verb|wait|:

\begin{itemize}
	\item \verb"active"/\verb"not_active";
	\item \verb"closed"/\verb"not_closed";
	\item \verb"id" и \verb"re_id";
	\item \verb"automation_id" и \verb"re_automation_id";
	\item \verb"title" и \verb"re_title".
\end{itemize}



\subsubsection*{Возвращаемое значение}
Если не задан один из флагов \verb"single" или \verb"none", всегда возвращает список найденных объектов <<окно>>. Если упомянутые флаги заданы, то правила таковы. \verb"none" заставляет вернуть булевское значение (\verb"True" --- окна не найдены, \verb"False" --- окна найдены), либо выбросить вместо возвращения \verb"False" исключение, если задан параметр \verb"assert". \verb"single" стремится вернуть объект <<окно>> или пустой объект \verb"None", но при наличии флага \verb"assert" вместо возвращения \verb"None" выбросит исключение.



\subsubsection*{Примеры}
Получив все окна верхнего уровня, отфильтруем список так, чтобы удалить из него окна с заголовком \verb"ipy64.exe":

\begin{verbatim}
@{wl}=    Wnd Get
@{wl}=    Wnd Filter    ${wl}    negative    title    ipy64.exe
\end{verbatim}


\subsection{Wnd Attr (работа с атрибутами окон)}
Оперирует атрибутами окон.


\subsubsection*{Название} 
\verb"Wnd Attr"

\subsubsection*{Позиционные параметры} 
\verb"li" -- список окон или один объект <<окно>>. В случае списка будут закрыты все.

\subsubsection*{Именованые параметры} 
\verb"assert" изменяет поведение стандартным для себя образом. Выбрасывает исключение в том случае, если есть операции ожидания, а дождаться не удалось.
\verb"timeout" указывает величину ожидания.

\subsubsection*{Поддерживаемые атрибуты} 

Поддерживаются все разрешённые для атрибутов действия:

\begin{itemize}
	\item \verb"active"/\verb"not_active";
	\item \verb"closed"/\verb"not_closed";
	\item \verb"id" и \verb"re_id";
	\item \verb"automation_id" и \verb"re_automation_id";
	\item \verb"title" и \verb"re_title";
	\item \verb"close" (действие \verb"do" закрывает окно).
\end{itemize}

\subsubsection*{Возвращаемое значение} 
Список результатов для каждого из объектов (или один такой результат, если на вход был подан объект, а не список объектов).

Для каждого из объектов соответствующий элемент списка --- результат операции \verb|get|, а если этих операций несколько --- список результатов (упорядоченный по имени атрибута).

\subsubsection*{Примеры}
\begin{verbatim}Wnd Attr    ${li}    do close\end{verbatim}





\section{Группа Ctl}
Ключевые слова группы \verb"Ctl" позволяют искать элементы GUI и оперировать ими.

\subsection{Ctl Get (поиск атрибутов)}
Это ключевое слово возвращает полный список объектов, описывающих процессы в системе.


\subsubsection*{Название} 
\verb"Ctl Get"

\subsubsection*{Позиционные параметры} 
Нет

\subsubsection*{Именованые параметры}

Стандартный набор, связанный с условиями завершения и ожиданием.

\subsubsection*{Поддерживаемые атрибуты} 

Полный набор атрибутов элементов GUI с действием \verb|wait|.

\subsubsection*{Возвращаемое значение} 
Список объектов <<процесс>>.

\subsubsection*{Примеры}
\begin{verbatim}@{pl}=    Proc List\end{verbatim}






\end{document}
